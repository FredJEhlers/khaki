\documentclass[11pt]{article}
\addtolength{\oddsidemargin}{-1.cm}
\addtolength{\textwidth}{2cm}
\addtolength{\topmargin}{-2cm}
\addtolength{\textheight}{3.5cm}

\usepackage[pdftex]{graphicx}
\usepackage{hyperref}
\usepackage{float}
\usepackage{cite}
\hypersetup{
	colorlinks=true,
	linkcolor=black,
	filecolor=magenta,
	urlcolor=cyan,
}

% define the title
\author{Team Kahki}
\title{Requirements Specification}

\begin{document}
	\setlength{\parskip}{6pt}
	
	% generates the title
	\begin{titlepage}
	
	\begin{center}
		% Upper part of the page       
		\includegraphics[width=0.7\linewidth]{Images/uniLogo.jpg}\\[1cm]    
		\textsc{\LARGE Khaki Round1}\\[0.3cm]
		\includegraphics[width=0.5\linewidth]{Images/TeamLogoSmall.jpg}\\[0.5cm]
		% Title
		\rule{\linewidth}{0.5mm} \\[1cm]
		{ \huge \bfseries Requirements Specification}\\[0.5cm]
		\rule{\linewidth}{0.5mm} \\[1cm] 			
		
		\begin{minipage}{0.4\textwidth}
			\begin{flushleft} \large
				Kulani {Bamuza}
			\end{flushleft}
		\end{minipage}
		\begin{minipage}{0.4\textwidth}
			\begin{flushright} \large
				\emph{} \\
				15008402 
			\end{flushright}
		\end{minipage}
		
		
		\begin{minipage}{0.4\textwidth}
			\begin{flushleft} \large
				\emph{} \\
				Frederick {Ehlers}
			\end{flushleft}
		\end{minipage}
		\begin{minipage}{0.4\textwidth}
			\begin{flushright} \large
				\emph{} \\
				11061112
			\end{flushright}
		\end{minipage}
		
		
		\begin{minipage}{0.4\textwidth}
			\begin{flushleft} \large
				\emph{} \\
				Dimpho {Mahoko}
			\end{flushleft}
		\end{minipage}
		\begin{minipage}{0.4\textwidth}
			\begin{flushright} \large
				\emph{} \\
				15175091
			\end{flushright}
		\end{minipage}

		\begin{minipage}{0.4\textwidth}
			\begin{flushleft} \large
				\emph{} \\
				Katlego {Mogokonyane}
			\end{flushleft}
		\end{minipage}
		\begin{minipage}{0.4\textwidth}
			\begin{flushright} \large
				\emph{} \\
				12134229
			\end{flushright}
		\end{minipage}

		\begin{minipage}{0.4\textwidth}
			\begin{flushleft} \large
				\emph{} \\
				Maria {Qumayo}
			\end{flushleft}
		\end{minipage}
		\begin{minipage}{0.4\textwidth}
			\begin{flushright} \large
				\emph{} \\
				29461775
			\end{flushright}
		\end{minipage}

		\begin{minipage}{0.4\textwidth}
			\begin{flushleft} \large
				\emph{} \\
				Craig van Heerden
			\end{flushleft}
		\end{minipage}
		\begin{minipage}{0.4\textwidth}
			\begin{flushright} \large
				\emph{} \\
				15029779
			\end{flushright}
		\end{minipage}

		\begin{minipage}{0.4\textwidth}
			\begin{flushleft} \large
				\emph{} \\
				Linda {Zwane}
			\end{flushleft}
		\end{minipage}
		\begin{minipage}{0.4\textwidth}
			\begin{flushright} \large
				\emph{} \\
				14199468
			\end{flushright}
		\end{minipage}
		
		\textsc{\Large Stakeholders}\\[1cm]	
		
		\begin{minipage}{0.4\textwidth}
			\begin{flushleft} \large
				\emph{} \\
				Computer Science Department of University of Pretoria
			\end{flushleft}
		\end{minipage}
		\begin{minipage}{0.4\textwidth}
			\begin{flushright} \large
				\emph{} \\
				Vreda Pieterse
			\end{flushright}
		\end{minipage}
		
	\end{center}
\end{titlepage}
	
	\tableofcontents
	
	\newpage
	\section{Introduction}
	The introduction of the Software Requirements Specification provides an overview of the entire specification with purpose, scope, definitions, acronyms, abbreviations, references and overview of the SRS. The aim of this document is to define the problem in detail and provide the detailed requirements for NavUP.
	
	\subsection{Purpose}
	The purpose of this SRS document is to provide a detailed description of NavUP by collecting and analyzing the ideas that define the system. This document describes NavUP’s user interface, External Interface, functional, and performance requirements. The document also describes the users of NavUp and its functions. The document helps developers of the NavUp system in software delivery lifecycle processes. 

	\subsection{Scope}
	The product as mentioned before is called NavUP, nav being an abbreviation for navigation and UP is an acronym for University of Pretoria.
	The product should be available on all major mobile operating systems to ensure most users can use the product.
	The basic functionality of the product should be similar to the basic functionalities of navigation systems like Google Maps and Waze. It should be able to provide the user with their current location, it should be able to to search for locations and venues, it should be able to provide the user with navigation to a location or venue, and it should be able to save locations or venues. 
	
	The system must be able to provide the user their location outdoors as well as indoors. GPS will therefore not suffice because the GPS receiver will not be able to receive a signal indoors. The system will therefore only use Wi-Fi and crowdsourcing to determine the user's location.
	
	The system should also have different levels of users, users with higher levels should be able to add new locations into the system. These locations can include points of interest, events and activities.
	
	The system should be able to give the user information about how busy certain areas of the campus are. This can shown to the user visually through a heat map.
	
	The system should also give users notifications based on their current location like points of interest. The system can learn what type of locations the user likes based on their previous locations and suggest them new locations to visit. The system should also record the user's movement data and reward them in a game like fashion. 
	
	\subsection{Definitions, Acronyms, and Abbreviations}
	\begin{table}[]
		\centering
		\resizebox{\textwidth}{!}{%
			\begin{tabular}{|l|l|}
				\hline
				GPS           & Global Positioning System. Used to determine a location on earth using satelite.                                     \\ \hline
				Accelerometer & An instrument for measuring the acceleration of a moving or vibrating body.                                          \\ \hline
				Heat map      & Visual dashboard that shows the concentration of subject of interest (pedestrian traffic) computed statistically.    \\ \hline
				Wi-Fi         & Wireless Fidelity. A form of wireless network communication.                                                         \\ \hline
				GUI           & Graphical User Interphase where a user can interact with the system by making inputs, searching and getting outputs. \\ \hline
			\end{tabular}%
		}
	\end{table}

	\subsection{References}
	\begin{itemize}
		\item Higher Specification
	\end{itemize}

	\subsection{Overview}
	The remaining sections of this document describes the context of the product, summary of the product’s functions, describes the characteristics of the users, outlines the restrictions of the solution space, lists the factors that affect the requirements, and it describes the software requirements including external interface requirements, functional requirements,  and performance requirements. Section 2 provides an overview of the product. Section 3 provides a detailed description for each of the system interfaces, provides a detailed description of the products functionality, describes all the performance related capabilities of the product and outlines all the restrictions. 

	\section{Overall Description}
	
		\subsection{Product Perspective}

		\subsection{Product Functions}

		\subsection{User Characteristics}

		\subsection{Constraints}

		\subsection{Assumptions and Dependencies}

	\section{Specific Requirements}
	
		\subsection{External Interface Requirements}

		\subsection{Functional Requirement}

			\begin{enumerate}
					\item User1 - The User
					\begin{itemize}
						\item stuff here
					\end{itemize}
			\end{enumerate}
		\subsection{Performance Requirements}
			\begin{enumerate}
				\item Low battery consumption
					\begin{itemize}
						\item Since the application uses predominantly Wifi will be priority to minimize impact on the battery of the phone since battery life is high concern of the users.   
						\item Numerous calls to various functions when navigating the user from point A to point B, thus optimising this functionality will be a great concern since this can become a high impact on the battery, draining it rather fast.
						\item With the added assumption that the user will grant permission to allow us transmit the device's location in attempt to help the application to navigate the user to locations with no signal for GPS but have WiFi accessibility.
						\item Heat map generation will require additional request sent from devices to the routers if this is constantly updated it will start impacting the battery heavily.
					\end{itemize}
				\item Fast real-time navigation updates
					\begin{itemize}
						\item Important aspect of navigating a user is the accuracy of the process. Thus making use of available resource effectively can create an accurate representation and navigation the user to the desired location. The resources that will be used is the Heat Map, other user locations, WiFi and GPS when available.
					\end{itemize}
				\item Accurate Step Counting
					\begin{itemize}
						\item Awards will be awarded based on steps taken thus the counting must use as much of the resources at its deposal to be as accurate as possible to distinguish between actual walking and people bouncing their legs when seated or swinging their phone around to attempt to fool the system.
					\end{itemize}
				\item Accurate location tracking
					\begin{itemize}
						\item This is the main promise of the application. Everything will be off and wrong if the location is inaccurate. Thus this performance requirement will be most important.
					\end{itemize}  
			\end{enumerate}
		\subsection{Design Constraints}

		\subsection{Software System Attributes}
			\begin{enumerate}
				\item Reliability 
					\begin{itemize}
						\item Our system makes a lot of assumptions in terms of certain resources that will be available along with permission to use them. Such as User permission to access their GPS/Location/WiFi/Accelerometer.
						\item Our system relies on availability of WiFi accros the campus of the University of Pretoria, that campus has power and or the generators work if there is no power.
					\end{itemize}
				\item Security 
					\begin{itemize}
						\item There are four levels of Users with each having similar and different rights. The differences must be sure to authenticate the user trying to execute the function. If the user is not authorised to do so they must be denied access or if they are authorised they must be allowed to continue.
					\end{itemize}
				\item Availability
					\begin{itemize}
						\item The avaibility is non existant if the user denies all of the permissions. If we do not have access to the device's GPS/Location/WiFi/Accelerometer none of the application's functionality will be able to work.
						\item Limited availability overall but the main functionality will be achieved with permission to use the user's Location and the user's WiFi.
						\item Most accurate functionality of the main functions can be achieved with permission to use Location/WiFi and GPS of the device and use the user's permission to send anonmous pings to generate heat maps and use other devices' location to navigate the user.
					\end{itemize}
			\end{enumerate}
		\subsection{Other Requirements}

	\section{Appendixes}


	
\end{document}