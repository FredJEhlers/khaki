\documentclass[11pt]{article}
\addtolength{\oddsidemargin}{-1.cm}
\addtolength{\textwidth}{2cm}
\addtolength{\topmargin}{-2cm}
\addtolength{\textheight}{3.5cm}

\usepackage[pdftex]{graphicx}
\usepackage{hyperref}
\usepackage{float}
\usepackage{cite}
\hypersetup{
	colorlinks=true,
	linkcolor=black,
	filecolor=magenta,
	urlcolor=cyan,
}

% define the title
\author{Team Kahki}
\title{Requirements Specification}

\begin{document}
	\setlength{\parskip}{6pt}
	
	% generates the title
	\begin{titlepage}
	
	\begin{center}
		% Upper part of the page       
		\includegraphics[width=0.7\linewidth]{Images/uniLogo.jpg}\\[1cm]    
		\textsc{\LARGE Khaki Round1}\\[0.3cm]
		\includegraphics[width=0.5\linewidth]{Images/TeamLogoSmall.jpg}\\[0.5cm]
		% Title
		\rule{\linewidth}{0.5mm} \\[1cm]
		{ \huge \bfseries Requirements Specification}\\[0.5cm]
		\rule{\linewidth}{0.5mm} \\[1cm] 			
		
		\begin{minipage}{0.4\textwidth}
			\begin{flushleft} \large
				Kulani {Bamuza}
			\end{flushleft}
		\end{minipage}
		\begin{minipage}{0.4\textwidth}
			\begin{flushright} \large
				\emph{} \\
				15008402 
			\end{flushright}
		\end{minipage}
		
		
		\begin{minipage}{0.4\textwidth}
			\begin{flushleft} \large
				\emph{} \\
				Frederick {Ehlers}
			\end{flushleft}
		\end{minipage}
		\begin{minipage}{0.4\textwidth}
			\begin{flushright} \large
				\emph{} \\
				11061112
			\end{flushright}
		\end{minipage}
		
		
		\begin{minipage}{0.4\textwidth}
			\begin{flushleft} \large
				\emph{} \\
				Dimpho {Mahoko}
			\end{flushleft}
		\end{minipage}
		\begin{minipage}{0.4\textwidth}
			\begin{flushright} \large
				\emph{} \\
				15175091
			\end{flushright}
		\end{minipage}

		\begin{minipage}{0.4\textwidth}
			\begin{flushleft} \large
				\emph{} \\
				Katlego {Mogokonyane}
			\end{flushleft}
		\end{minipage}
		\begin{minipage}{0.4\textwidth}
			\begin{flushright} \large
				\emph{} \\
				12134229
			\end{flushright}
		\end{minipage}

		\begin{minipage}{0.4\textwidth}
			\begin{flushleft} \large
				\emph{} \\
				Maria {Qumayo}
			\end{flushleft}
		\end{minipage}
		\begin{minipage}{0.4\textwidth}
			\begin{flushright} \large
				\emph{} \\
				29461775
			\end{flushright}
		\end{minipage}

		\begin{minipage}{0.4\textwidth}
			\begin{flushleft} \large
				\emph{} \\
				Craig van Heerden
			\end{flushleft}
		\end{minipage}
		\begin{minipage}{0.4\textwidth}
			\begin{flushright} \large
				\emph{} \\
				15029779
			\end{flushright}
		\end{minipage}

		\begin{minipage}{0.4\textwidth}
			\begin{flushleft} \large
				\emph{} \\
				Linda {Zwane}
			\end{flushleft}
		\end{minipage}
		\begin{minipage}{0.4\textwidth}
			\begin{flushright} \large
				\emph{} \\
				14199468
			\end{flushright}
		\end{minipage}
		
		\textsc{\Large Stakeholders}\\[1cm]	
		
		\begin{minipage}{0.4\textwidth}
			\begin{flushleft} \large
				\emph{} \\
				Computer Science Department of University of Pretoria
			\end{flushleft}
		\end{minipage}
		\begin{minipage}{0.4\textwidth}
			\begin{flushright} \large
				\emph{} \\
				Vreda Pieterse
			\end{flushright}
		\end{minipage}
		
	\end{center}
\end{titlepage}
	
	\tableofcontents
	
	\newpage
	\section{Introduction}
	The introduction of the Software Requirements Specification provides an overview of the entire specification with purpose, scope, definitions, acronyms, abbreviations, references and overview of the SRS. The aim of this document is to define the problem in detail and provide the detailed requirements for NavUP.
	
	\subsection{Purpose}
	The purpose of this SRS document is to provide a detailed description of NavUP by collecting and analyzing the ideas that define the system. This document describes NavUP’s user interface, External Interface, functional, and performance requirements. The document also describes the users of NavUp and its functions. The document helps developers of the NavUp system in software delivery lifecycle processes. 

	\subsection{Scope}
	The product as mentioned before is called NavUP, nav being an abbreviation for navigation and UP is an acronym for University of Pretoria.
	The product should be available on all major mobile operating systems to ensure most users can use the product.
	The basic functionality of the product should be similar to the basic functionalities of navigation systems like Google Maps and Waze. It should be able to provide the user with their current location, it should be able to to search for locations and venues, it should be able to provide the user with navigation to a location or venue, and it should be able to save locations or venues. 
	
	The system must be able to provide the user their location outdoors as well as indoors. GPS will therefore not suffice because the GPS receiver will not be able to receive a signal indoors. The system will therefore only use Wi-Fi and crowdsourcing to determine the user's location.
	
	The system should also have different levels of users, users with higher levels should be able to add new locations into the system. These locations can include points of interest, events and activities.
	
	The system should be able to give the user information about how busy certain areas of the campus are. This can shown to the user visually through a heat map.
	
	The system should also give users notifications based on their current location like points of interest. The system can learn what type of locations the user likes based on their previous locations and suggest them new locations to visit. The system should also record the user's movement data and reward them in a game like fashion. 
	
	\subsection{Definitions, Acronyms, and Abbreviations}
	\begin{table}[]
		\centering
		\begin{tabular}{|l|l|}
			\hline
			GPS           & Global Positioning System. Used to determine a location on earth using satelite.                                     \\ \hline
			Accelerometer & An instrument for measuring the acceleration of a moving or vibrating body.                                          \\ \hline
			Heat map      & Visual dashboard that shows the concentration of subject of interest (pedestrian traffic) computed statistically.    \\ \hline
			Wi-Fi         & Wireless Fidelity. A form of wireless network communication.                                                         \\ \hline
			GUI           & Graphical User Interphase where a user can interact with the system by making inputs, searching and getting outputs. \\ \hline
		\end{tabular}
	\end{table}

	\subsection{References}
	\begin{itemize}
		\item Higher Specification
	\end{itemize}

	\section{Functional Requirements}
	
	
\end{document}