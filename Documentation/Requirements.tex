\documentclass[11pt]{article}
\addtolength{\oddsidemargin}{-1.cm}
\addtolength{\textwidth}{2cm}
\addtolength{\topmargin}{-2cm}
\addtolength{\textheight}{3.5cm}

\usepackage[pdftex]{graphicx}
\usepackage{hyperref}
\usepackage{float}
\usepackage{cite}
\hypersetup{
	colorlinks=true,
	linkcolor=black,
	filecolor=magenta,
	urlcolor=cyan,
}

% define the title
\author{Team Kahki}
\title{Requirements Specification}

\begin{document}
	\setlength{\parskip}{6pt}
	
	% generates the title
	\begin{titlepage}
	
	\begin{center}
		% Upper part of the page       
		\includegraphics[width=0.7\linewidth]{Images/uniLogo.jpg}\\[1cm]    
		\textsc{\LARGE Khaki Round1}\\[0.3cm]
		\includegraphics[width=0.5\linewidth]{Images/TeamLogoSmall.jpg}\\[0.5cm]
		% Title
		\rule{\linewidth}{0.5mm} \\[1cm]
		{ \huge \bfseries Requirements Specification}\\[0.5cm]
		\rule{\linewidth}{0.5mm} \\[1cm] 			
		
		\begin{minipage}{0.4\textwidth}
			\begin{flushleft} \large
				Kulani {Bamuza}
			\end{flushleft}
		\end{minipage}
		\begin{minipage}{0.4\textwidth}
			\begin{flushright} \large
				\emph{} \\
				15008402 
			\end{flushright}
		\end{minipage}
		
		
		\begin{minipage}{0.4\textwidth}
			\begin{flushleft} \large
				\emph{} \\
				Frederick {Ehlers}
			\end{flushleft}
		\end{minipage}
		\begin{minipage}{0.4\textwidth}
			\begin{flushright} \large
				\emph{} \\
				11061112
			\end{flushright}
		\end{minipage}
		
		
		\begin{minipage}{0.4\textwidth}
			\begin{flushleft} \large
				\emph{} \\
				Dimpho {Mahoko}
			\end{flushleft}
		\end{minipage}
		\begin{minipage}{0.4\textwidth}
			\begin{flushright} \large
				\emph{} \\
				15175091
			\end{flushright}
		\end{minipage}

		\begin{minipage}{0.4\textwidth}
			\begin{flushleft} \large
				\emph{} \\
				Katlego {Mogokonyane}
			\end{flushleft}
		\end{minipage}
		\begin{minipage}{0.4\textwidth}
			\begin{flushright} \large
				\emph{} \\
				12134229
			\end{flushright}
		\end{minipage}

		\begin{minipage}{0.4\textwidth}
			\begin{flushleft} \large
				\emph{} \\
				Maria {Qumayo}
			\end{flushleft}
		\end{minipage}
		\begin{minipage}{0.4\textwidth}
			\begin{flushright} \large
				\emph{} \\
				29461775
			\end{flushright}
		\end{minipage}

		\begin{minipage}{0.4\textwidth}
			\begin{flushleft} \large
				\emph{} \\
				Craig van Heerden
			\end{flushleft}
		\end{minipage}
		\begin{minipage}{0.4\textwidth}
			\begin{flushright} \large
				\emph{} \\
				15029779
			\end{flushright}
		\end{minipage}

		\begin{minipage}{0.4\textwidth}
			\begin{flushleft} \large
				\emph{} \\
				Linda {Zwane}
			\end{flushleft}
		\end{minipage}
		\begin{minipage}{0.4\textwidth}
			\begin{flushright} \large
				\emph{} \\
				14199468
			\end{flushright}
		\end{minipage}
		
		\textsc{\Large Stakeholders}\\[1cm]	
		
		\begin{minipage}{0.4\textwidth}
			\begin{flushleft} \large
				\emph{} \\
				Computer Science Department of University of Pretoria
			\end{flushleft}
		\end{minipage}
		\begin{minipage}{0.4\textwidth}
			\begin{flushright} \large
				\emph{} \\
				Vreda Pieterse
			\end{flushright}
		\end{minipage}
		
	\end{center}
\end{titlepage}
	
	\tableofcontents
	\newpage
		
	\section{Functional Requirements}
	
	\section {2 Overview}
	
	\subsection{2.2 Product Function}
		The application will be able to track a user’s location. The user will search for a venue on campus by room number and 			be directed to that venue based on distance, time and crowd traffic. The app can also add a waypoint if the user wants 			to make a stop before the final destination. The application will also track the user’s steps and notify and reward them 		 through a third party when they reach milestones. NavUP will also provide the ability for higher level users to push 			notifications to users if there are any events that user might be interested in. If the user isn’t interested, they can 		block the notification resulting in the application learning their preferences. 

	\subsection{2.3 User characteristics}
		There are 4 user profiles available on the application. A guest user who is a user that isn’t a frequent campus visitor. 		 For instance a visitor to a staff member, someone coming for a conference or meeting. This user will only be able to 			access a limited amount of functionality like location services and step counting.  

		A registered user would typically be a staff member or a student. Someone who’s on campus frequently and they will be 			able to access most of the functionality excluding higher level and third party functionality. So all the guest 			functionality with the added ability to add favourite venues and get rewards based on the amount of steps they’ve taken 		relative to time. They would also receive notifications based on their preferences.

		A higher level user would be an organization like a society, a political party, a faculty etc.They can add events to the 		 pushed notifications and add new venues onto the app. 

		Third party user will be used for registered user reward givers like stores in and around campus. They will be able to 			indicate what rewards they have available, the system will reward at it’s own discretion and the third party will 			recieve a way to validate which user gets what reward.

		\subsection{2.4 Constraints}
		The application will not be able to fully function without a connection to the internet Because it will need to extract 		data from the server.

		The smart device being used to access the application must have internet connectivity, a GPS 
		System and WiFi.

		The application is going to be constrained by the fact that it needs to access the devices’ 
		GPS system. The fact that there are different devices with different manufacturers, the Interface between the different 		devices will differ.

		\subsection{2.5 Assumptions and dependencies}
		It is assumed that the applications will be used on either a smartphone or a smart tablet with location services, WiFi 			and mobile data enabled.
		The device is assumed to have enough memory to handle the application
		It is also assumed that the accelerometer hardware is installed on the device and well in function.
		It is assumed that your phone has enough battery to handle the application throughout the day
		The application’s performance is dependent on that of the device.
		The accuracy of the step counter is dependent on the device accelerometer
		
		\section{3 Specific Requirements}
		
		\subsection{3.4 Design Constraints}
		The database has to be able to handle more than 30000 users. The applications is supposed to be designed in such a way 			that it can work with WiFi, GPS and Mobile networks. It should switch between WiFi as necessary.

		The application will need hard drive space on the device. The lack thereof will result in an inability to download. 			Based on research the application should need between 100 - 200 MB of harddrive space. 

		Memory usage will be high because the application will always be running in order to track the user’s steps unless the 			user decides otherwise. The data should not be more than 20MB.


	
	
\end{document}
