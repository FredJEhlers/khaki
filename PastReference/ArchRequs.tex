\documentclass[a4paper,12pt,titlepage]{article}
\usepackage{graphicx}
\usepackage[hidelinks]{hyperref}
\usepackage{listings}
%\setcounter{tocdepth}{1}
\usepackage{float}
\usepackage[T1]{fontenc}




\DeclareGraphicsExtensions{.png}
%\graphicspath{ {./images/} }

\begin{document}
\input{./titlepage.tex}


%-----------------TABLE OF CONTENT-----------------
\newpage
\tableofcontents


%-----------------BASIC INTRODUCTION-----------------
\newpage
\section{Introduction}

This document was compiled by our group during our meetings and was produced as a whole by the team. \bigskip

This document contains specifications of the software architecture requirements. This is the infrastructure upon which the application functionality will be developed. \bigskip

The following non-functional requirements are addressed in depth with supporting diagrams(when necessary):

\begin{itemize}
	\item Quality requirements.
	\item Integration and access channel requirements.
	\item Architectural constraints
	\item Architectural patterns or styles
	\item Architectural tactics or strategies
	\item Use of reference architectures and frameworks
	\item Access and integration channels
	\item Technologies
\end{itemize}

%-----------------PARTS TO EDIT-----------------
\newpage

\section{Architecture Requirements}
\subsection{Architectural Scope} %Maria
\input{Arch_Scope}
\newpage
\subsection{Critical Quality Requirements}
\subsubsection{Scalability}%Ruth
	\input{Scalability}
\subsubsection{Security}%Kabelo
	\input{Security}
\subsubsection{Usability}%Lindelo
	\subsubsection*{Description}
	Buzz should be easy to use and the user should be able to remember how Buzz works; despite their long absence. The interface should appear easy to use; the visuals should not frustrate the user.
	\subsubsection*{Justification}
	Usability is critical because Buzz is user-oriented and requires that users don't get frustrated with the interface. 
	\subsubsection*{Mechanism}
	\begin{enumerate}
		\item Strategy:
		\begin{itemize}
			\item Descriptive headings that will not be ambiguous to the user.
			\item Clear and descriptive "help" messages that will initiate the user into Buzz. 
			\item Descriptive error messages that will tell the user an error has occurred and the necessary steps to address the errors.
			\item Grouped content; this will make the interface look easier to use. If a user is looking for content, they should find related content in the same place. Navigation becomes easier.
			\item A user should be given feedback on their actions. If a page is still loading, for example, the user should be aware that the page is loading.
		\end{itemize} 
		\item Architectural Pattern(s):
		\begin{itemize}
		\item Model-View-Controller: The user will only be presented with the view, which means they won't directly interact with other components of the system. This simplifies the user's interaction with Buzz.
		\item Layering is similar to Model-View-Controller; the user only interacts with one layer and that layer passes the requests to a lower layer such that the user is oblivious of the complexity of low-level layers.
		\end{itemize} 
	\end{enumerate}
\subsubsection{Integrability}%Sylvester
	\input{Integrability_Quality_Requirement}
	
\newpage
\subsection{Important Quality Requirements}

\subsubsection{Performance}%Thinus
	\input{Performance}
	\newpage
\subsubsection{Plug-ability (Maintainability)}%Roger
	\input{./plug-ability.tex}
\newpage
\subsubsection{Monitor-ability}%Maria
\input{Monitor-abiliy_Auditorabiliy}

\newpage
\subsection{Nice to have quality requirements}
\subsubsection{Reliability and Availability}%Axel
\input{Relaiability}
	\newpage
\subsubsection{Testability}%Maphuti
\input{testability}
\newpage
\subsection{Architectural Constraints}
	\begin{itemize}
		%\item\textbf{Neo4j}, the preferable database technology, is schema-less, which means a schema change will not affect how the data is stored. It includes other features that would improve Buzz, but JQPL has to be used. \\ 
		\item\textbf{JPQL} was chosen by the client, although Neo4j would have been better. 
		\item\textbf{Java EE} is the platform that will be used to implement Buzz, the system is constrained to this platform.
		\item\textbf{JSF}, was chosen by the client and forms part of Java EE. 
		\item\textbf{AJAX} the client requested that this technology be used. 
%<<<<<<< HEAD
		\item\textbf{HTML} is the only markup language that can be used as specified by the client.
		%\item\textbf{The operating system} that will run the Buzz server is Linux, as requested by the client.
%=======
		\item\textbf{The operating system} that will run the Buzz server is Linux, as requested by the client. We recommend Debian as the Linux distribution to use because:  
		\begin{itemize}
			\item Very complete.
			\item Supportive and active community 
			\item Multi-arch/kernel support
			\item Very stable and allows you your freedom
		\end{itemize}
%>>>>>>> origin/master
		\end{itemize}
\newpage
\section{Architectural Patterns or Styles}%Axel
\input{Patterns_and_Styles}
\newpage
\section{Architectural Tactics or Strategies}%Kabelo
	\input{Tactics_or_Strategies}
\newpage
\section{Use of Reference Architectures and Frameworks}%Thinus
A reference architecture provides a template solution for an architecture for a particular domain. We will use as our reference architecture the Java Platform, Enterprise Edition (Java EE) and Model View Controller framework for the Buzz Space system. 
Java EE includes an API and runtime environment for developing and running \textit{large-scale, multi-tiered, scalable, reliable, and secure network applications}.

Java EE will be used to implement a tiered architectural structure, each tier or layer containing one ore more of the Buzz Space modules in conjunction with Java EE modules. 

A multi-tiered application is thus an application where \textit{the functionality of the application is separated into isolated functional areas, called tiers}. An example of this can be seen in figure~\ref{fig:refarchandmvc} on page~\pageref{fig:refarchandmvc}.
\begin{figure}[H]
	\centering
	\fbox{\includegraphics[width=0.85\textwidth]{ReferenceArchitecture}}
	\caption{A basic overview of the buzz space tiered/layered application.}
	\label{fig:refarchandmvc}
\end{figure}
\newpage
\section{Access and Integration Channels}
\subsection{Access Channels}
\subsubsection{Human Access Channels}%Maria
\input{./Human-Access-Channels.tex}
\subsubsection{System Access Channels}%Roger
	\input{./System-Access-Channels.tex}
\subsection{Integration Channels}%Sylvester
	\input{./Integration_Requirements_1.tex}
\newpage
\section{Technologies}%Lindelo and Ruth
\input{Technologies}

\end{document}
